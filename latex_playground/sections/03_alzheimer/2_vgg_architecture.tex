% -----------------------------------------------------------------------------
% Description: 3-Phase Transfer Learning (Robust Fix for LR mode & Colors)
% -----------------------------------------------------------------------------
% -----------------------------------------------------------------------------
% -----------------------------------------------------------------------------

\begin{frame}{VGG16 Architecture: Tensor Flow Review}
    \centering
    \vspace{-0.2cm}

    % Resize to fit the slide perfectly
    \resizebox{\textwidth}{!}{%
    \begin{tikzpicture}[
        font=\sffamily\scriptsize,
        node distance=0.3cm,
        % Styles for layers
        conv/.style={
            rectangle, draw=red!80!black, fill=red!30, thick,
            minimum width=0.4cm, % Channel depth (Visual)
            inner sep=0pt
        },
        pool/.style={
            rectangle, draw=blue!80!black, fill=blue!30, thick,
            minimum width=0.2cm,
            inner sep=0pt
        },
        fc/.style={
            rectangle, draw=orange!80!black, fill=orange!30, thick,
            minimum height=0.5cm, minimum width=2.5cm,
            inner sep=2pt
        },
        label/.style={
            font=\tiny, align=center, text=black!70
        },
        arrow/.style={->, >=stealth, thick, black!60}
    ]

        % --- BLOCK 1: 224x224x64 ---
        % Input Image
        \node[inner sep=0pt] (input) {\includegraphics[width=1.5cm, height=1.5cm]{alzheimer/data_preprocessing/step6_roi}};
        \node[below=0.1cm of input, label] {$224 \times 224 \times 3$};

        % Conv 1 (Tall, Thin)
        \node[conv, right=0.5cm of input, minimum height=3.5cm] (c1) {};
        \node[conv, right=0.05cm of c1, minimum height=3.5cm] (c1b) {};
        \node[label, above=0.1cm of c1] {Conv 1\\$64$};

        % Pool 1
        \node[pool, right=0.05cm of c1b, minimum height=1.75cm] (p1) {}; % Half height
        \node[label, below=0.1cm of p1] {Pool\\/2};

        % --- BLOCK 2: 112x112x128 ---
        % Conv 2 (Shorter, Thicker)
        \node[conv, right=0.4cm of p1, minimum height=1.75cm, minimum width=0.6cm] (c2) {};
        \node[conv, right=0.05cm of c2, minimum height=1.75cm, minimum width=0.6cm] (c2b) {};
        \node[label, above=0.1cm of c2] {Conv 2\\$128$};

        % Pool 2
        \node[pool, right=0.05cm of c2b, minimum height=0.87cm] (p2) {};
        \node[label, below=0.1cm of p2] {Pool\\/2};

        % --- BLOCK 3: 56x56x256 ---
        \node[conv, right=0.4cm of p2, minimum height=0.87cm, minimum width=0.9cm] (c3) {};
        \node[conv, right=0.05cm of c3, minimum height=0.87cm, minimum width=0.9cm] (c3b) {};
        \node[conv, right=0.05cm of c3b, minimum height=0.87cm, minimum width=0.9cm] (c3c) {};
        \node[label, above=0.1cm of c3b] {Conv 3\\$256$};

        % Pool 3
        \node[pool, right=0.05cm of c3c, minimum height=0.43cm] (p3) {};

        % --- BLOCK 4: 28x28x512 ---
        \node[conv, right=0.4cm of p3, minimum height=0.43cm, minimum width=1.3cm] (c4) {};
        \node[conv, right=0.05cm of c4, minimum height=0.43cm, minimum width=1.3cm] (c4b) {};
        \node[conv, right=0.05cm of c4b, minimum height=0.43cm, minimum width=1.3cm] (c4c) {};
        \node[label, above=0.1cm of c4b] {Conv 4\\$512$};

        % Pool 4
        \node[pool, right=0.05cm of c4c, minimum height=0.21cm] (p4) {};

        % --- BLOCK 5: 14x14x512 ---
        \node[conv, right=0.4cm of p4, minimum height=0.21cm, minimum width=1.3cm] (c5) {};
        \node[conv, right=0.05cm of c5, minimum height=0.21cm, minimum width=1.3cm] (c5b) {};
        \node[conv, right=0.05cm of c5b, minimum height=0.21cm, minimum width=1.3cm] (c5c) {};
        \node[label, above=0.1cm of c5b] {Conv 5\\$512$};

        % Pool 5 (Output 7x7)
        \node[pool, right=0.05cm of c5c, minimum height=0.1cm] (p5) {};
        \node[label, below=0.1cm of p5] {$7 \times 7$};

        % --- CLASSIFIER HEAD (Flattened) ---
        \node[fc, right=0.6cm of p5, rotate=90] (fc1) {\tiny FC-4096};
        \node[fc, right=0.2cm of fc1, rotate=90] (fc2) {\tiny FC-4096};
        \node[fc, right=0.2cm of fc2, rotate=90, fill=green!30, draw=green!80!black, minimum width=1.5cm] (sm) {\tiny Softmax-1000};

        % --- CONNECTIONS ---
        \draw[arrow] (input) -- (c1);
        \draw[arrow] (p1) -- (c2);
        \draw[arrow] (p2) -- (c3);
        \draw[arrow] (p3) -- (c4);
        \draw[arrow] (p4) -- (c5);
        \draw[arrow] (p5) -- (fc1);

        % Legend
        \node[anchor=south east] at (current bounding box.south east) {
            \tiny
            \begin{tikzpicture}
                \node[fill=red!30, draw=red!80!black, minimum width=0.3cm] (l1) {}; \node[right=0.1cm of l1] {Conv+ReLU};
                \node[fill=blue!30, draw=blue!80!black, minimum width=0.3cm, right=1.8cm of l1] (l2) {}; \node[right=0.1cm of l2] {Max Pool};
                \node[fill=orange!30, draw=orange!80!black, minimum width=0.3cm, right=3.5cm of l1] (l3) {}; \node[right=0.1cm of l3] {Fully Connected};
            \end{tikzpicture}
        };

    \end{tikzpicture}
    }

    \vspace{0.3cm}
    \footnotesize \textit{Visualizing the reduction in spatial dimensions and increase in depth.}
\end{frame}


\begin{frame}{Progressive Transfer Learning Strategy}
    \centering

    % --- 1. DYNAMIC STYLE DEFINITIONS (Safe Method) ---
    % Define styles globally first to avoid undefined errors
    \tikzset{
        baseblock/.style={
            rectangle, draw=gray, thick, rounded corners=2pt,
            minimum height=2.5cm, minimum width=1.4cm,
            align=center, drop shadow={opacity=0.3}, font=\sffamily\footnotesize
        },
        frozen/.style={
            baseblock, top color=blue!10, bottom color=blue!5, draw=blue!50,
            label={[blue, font=\bfseries\tiny]south:[FROZEN]}
        },
        train/.style={
            baseblock, top color=red!10, bottom color=red!5, draw=red!50,
            label={[red, font=\bfseries\tiny]south:[TRAINABLE]}
        }
    }

    % Switch styles based on slide number using \only
    % This prevents parsing errors inside the node options
    \only<1-2>{\tikzset{styleB123/.style={frozen}}}
    \only<3->{\tikzset{styleB123/.style={train}}}

    \only<1>{\tikzset{styleB45/.style={frozen}}}
    \only<2->{\tikzset{styleB45/.style={train}}}

    % --- 2. ARCHITECTURE DIAGRAM ---
    % NO BLANK LINES allowed inside resizebox!
    \resizebox{0.95\textwidth}{!}{%
    \begin{tikzpicture}[node distance=0.3cm and 0.8cm, arrow/.style={->, >=stealth, thick, gray}]
        % 1. Input
        \node[draw=none] (input) {\includegraphics[width=1.5cm]{alzheimer/data_preprocessing/step6_roi}};
        \node[below=0.1cm of input, font=\scriptsize] {ROI Slice};
        % 2. Early Layers (Dynamic Style)
        \node[styleB123, right=of input, minimum height=3cm] (b123) {Early Layers\\(Block 1-3)\\ \scriptsize \textit{Low-level}};
        % 3. Deep Layers (Dynamic Style)
        \node[styleB45, right=of b123, minimum height=2.5cm] (b45) {Deep Layers\\(Block 4-5)\\ \scriptsize \textit{Abstract}};
        % 4. Head (Always Train)
        \node[train, right=of b45, minimum height=2cm, minimum width=2cm] (head) {Custom Head\\ \scriptsize \textit{Flatten+Dense}\\ \scriptsize \textit{Sigmoid}};
        % 5. Output
        \node[right=of head, circle, draw=black, thick, fill=green!10] (out) {P};
        \node[below=0.1cm of out, font=\scriptsize] {AD Prob};
        % Connections
        \draw[arrow] (input) -- (b123);
        \draw[arrow] (b123) -- (b45);
        \draw[arrow] (b45) -- (head);
        \draw[arrow] (head) -- (out);
    \end{tikzpicture}%
    }

    \vspace{0.5cm}

    % --- 3. DYNAMIC INFO CARD (Safe Colors) ---
    \begin{overlayarea}{\textwidth}{3.5cm}
        % Phase 1
        \only<1>{%
            \setbeamercolor{coloredbox}{bg=gray!10, fg=black}%
            \begin{beamercolorbox}[sep=1em, rounded=true, shadow=true]{coloredbox}
                \textbf{\Large Phase 1: Feature Extraction} \hfill \textit{Baseline}
                \par\noindent\rule{\textwidth}{0.4pt}
                \begin{columns}
                    \begin{column}{0.45\textwidth}
                        \begin{itemize}
                            \item \textbf{Strategy:} Freeze Backbone
                            \item \textbf{Target:} Train Head Only
                        \end{itemize}
                    \end{column}
                    \begin{column}{0.45\textwidth}
                        \begin{itemize}
                            \item \textbf{LR }$\eta=$ $10^{-4}$
                            \item \textbf{Epochs:} 10
                        \end{itemize}
                    \end{column}
                \end{columns}
            \end{beamercolorbox}%
        }
        % Phase 2
        \only<2>{%
            \setbeamercolor{coloredbox}{bg=yellow!10, fg=black}%
            \begin{beamercolorbox}[sep=1em, rounded=true, shadow=true]{coloredbox}
                \textbf{\Large Phase 2: Shallow Fine-tuning} \hfill \textit{Refinement}
                \par\noindent\rule{\textwidth}{0.4pt}
                \begin{columns}
                    \begin{column}{0.45\textwidth}
                        \begin{itemize}
                            \item \textbf{Strategy:} Unfreeze Last 8 Layers
                            \item \textbf{Target:} Adapt Abstract Feats
                        \end{itemize}
                    \end{column}
                    \begin{column}{0.45\textwidth}
                        \begin{itemize}
                            \item \textbf{LR }$\eta=$ $10^{-5}$ (\textcolor{red}{Reduced})
                            \item \textbf{Epochs:} 10
                        \end{itemize}
                    \end{column}
                \end{columns}
            \end{beamercolorbox}%
        }
        % Phase 3
        \only<3>{%
            \setbeamercolor{coloredbox}{bg=red!10, fg=black}%
            \begin{beamercolorbox}[sep=1em, rounded=true, shadow=true]{coloredbox}
                \textbf{\Large Phase 3: Deep Fine-tuning} \hfill \textit{Deep Adaptation}
                \par\noindent\rule{\textwidth}{0.6pt}
                \begin{columns}
                    \begin{column}{0.5\textwidth}
                        \begin{itemize}
                            \item \textbf{Strategy:} Unfreeze Last 10 Layers
                            \item \textbf{Target:} Full MRI Adaptation
                        \end{itemize}
                    \end{column}
                    \begin{column}{0.45\textwidth}
                        \begin{itemize}
                            \item \textbf{LR }$\eta=$ $10^{-5}$
                            \item \textbf{Epochs:} 20
                        \end{itemize}
                    \end{column}
                \end{columns}
            \end{beamercolorbox}%
        }
    \end{overlayarea}
\end{frame}

\begin{frame}{Subject-Level Inference Strategy}
    \centering
    \vspace{-0.2cm}

    % --- PART 1: MATH LOGIC (Color-Coded) ---
    \begin{beamercolorbox}[sep=5pt, center, rounded=true, shadow=false, bg=blue!5]{math}
        \small
        \textbf{Logic:} Aggregate slice predictions ($p_i$) to diagnose the \textbf{patient} ($P_{subject}$).
        \par\noindent\rule{0.8\textwidth}{0.4pt}
        \vspace{0.1cm}

        \begin{columns}
            \begin{column}{0.45\textwidth}
                \centering
                % Color coding: P_subject (Violet), p1, p2 (Blue)
                $\textcolor{violet}{P_{subject}} = \frac{\textcolor{DeepBlue}{p_1} + \textcolor{DeepBlue}{p_2}}{2}$
            \end{column}
            \begin{column}{0.05\textwidth}
                \centering \textbf{$\Rightarrow$}
            \end{column}
            \begin{column}{0.45\textwidth}
                \centering
                $\hat{y} = \begin{cases}
                    \text{AD} & \text{if } \textcolor{violet}{P_{subject}} > 0.5 \\
                    \text{CN} & \text{otherwise}
                \end{cases}$
            \end{column}
        \end{columns}
    \end{beamercolorbox}

    \vspace{0.2cm}

    % --- PART 2: HORIZONTAL FLOWCHART (Linked Variables) ---
    \resizebox{\textwidth}{!}{%
    \begin{tikzpicture}[
        node distance=0.5cm and 0.6cm,
        font=\sffamily\small,
        % Styles
        entity/.style={
            rectangle, draw=DeepBlue, thick, fill=white,
            minimum width=2.0cm, minimum height=1cm, rounded corners=3pt,
            drop shadow
        },
        process/.style={
            rectangle, draw=gray, thick, fill=gray!10,
            minimum width=2.8cm,
            minimum height=0.8cm, rounded corners=1pt
        },
        decision/.style={
            diamond, draw=red!70, thick, fill=red!5,
            aspect=2, inner sep=1pt, font=\footnotesize\bfseries
        },
        arrow/.style={->, >=stealth, ultra thick, DeepBlue, rounded corners=5pt}
    ]

        % 1. START
        \node[entity, fill=blue!10] (patient) {\textbf{Patient X}};

        % 2. BRANCHING
        \node[process, above right=0.3cm and 0.8cm of patient] (s1) {Slice 1};
        \node[process, below right=0.3cm and 0.8cm of patient] (s2) {Slice 2};

        % 3. INFERENCE
        \node[process, right=0.5cm of s1, fill=yellow!10] (vgg1) {Fused VGG16($s_1$)};
        \node[process, right=0.5cm of s2, fill=yellow!10] (vgg2) {Fused VGG16($s_2$)};

        % 4. PROBABILITIES (Linked to Math)
        % Explicitly labeling p1 and p2 with DeepBlue color
        \node[right=0.2cm of vgg1, text=DeepBlue, font=\bfseries] (p1) {$p_1 = 0.8$};
        \node[right=0.2cm of vgg2, text=DeepBlue, font=\bfseries] (p2) {$p_2 = 0.4$};

        % 5. AGGREGATION
        \coordinate (midpoint) at ($(p1)!0.5!(p2)$);
        \node[circle, draw=DeepBlue, ultra thick, right=1.2cm of midpoint] (plus) {\Large $\Sigma$};
        \node[below=0.1em of plus, font=\tiny] {AVG};

        % 6. DECISION (Using P_subject)
        \node[decision, right=1.5cm of plus] (dec) {$0.6 > 0.5$ ?};

        % Label for P_subject on the arrow (Violet)
        \path (plus) -- node[above, font=\scriptsize, text=violet, yshift=2pt] {$P_{subject}=0.6$} (dec);

        % 7. OUTCOMES (Spaced out for Yes/No)
        % Move outcomes further right to allow clean wiring
        \node[entity, above right=0.1cm and 1.8cm of dec, draw=red!80, fill=red!10] (ad) {\textbf{AD} ($\hat{y}=1$)};
        \node[entity, below right=0.1cm and 1.8cm of dec, draw=green!60!black, fill=green!10] (cn) {\textbf{CN} ($\hat{y}=0$)};

        % --- WIRING ---

        % Patient -> Slices
        \draw[arrow] (patient.east) -- ++(0.4,0) |- (s1.west);
        \draw[arrow] (patient.east) -- ++(0.4,0) |- (s2.west);

        % Slices -> Model -> Prob
        \draw[arrow] (s1) -- (vgg1) -- (p1);
        \draw[arrow] (s2) -- (vgg2) -- (p2);

        % Prob -> Sigma
        \draw[arrow] (p1.east) -- ++(0.2,0) |- (plus.west);
        \draw[arrow] (p2.east) -- ++(0.2,0) |- (plus.west);

        % Sigma -> Decision
        \draw[arrow] (plus) -- (dec);

        % Decision -> Outcomes (Clean Split)
        % Create a split point to separate the branching from the diamond
        \draw[arrow] (dec.east) -- ++(0.5, 0) coordinate(split);

        % YES Branch (Up) - Label clearly above the line
        \draw[arrow] (split) |- node[pos=0.7, above, text=red, font=\bfseries\scriptsize, yshift=0pt] {Yes} (ad.west);

        % NO Branch (Down) - Label clearly below the line
        \draw[arrow, dashed, gray] (split) |- node[pos=0.7, below, text=green!60!black, font=\bfseries\scriptsize, yshift=0pt] {No} (cn.west);

    \end{tikzpicture}
    }
\end{frame}