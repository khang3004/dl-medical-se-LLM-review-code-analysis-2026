% --- SLIDE 1: THE VISUAL HOOK (Animated) ---
\begin{frame}{The Silent Crisis: Alzheimer's Disease (AD)}
    \begin{columns}
        % Left Column: Image (Always visible to anchor the slide)
        \begin{column}{0.5\textwidth}
            \centering
            \begin{figure}
                \includegraphics[width=0.9\textwidth, height=0.6\textheight, keepaspectratio]{alzheimer/ad_brain_comparison}
                \caption{\footnotesize Healthy Brain vs. Severe AD Brain. Note the extreme shrinkage (atrophy).}
            \end{figure}
        \end{column}

        % Right Column: Stats (Cascading Animation)
        \begin{column}{0.5\textwidth}
            \onslide<2->{
                \textbf{Why this matters NOW?}
                \vspace{0.5cm}
            }

            % --- Stat Box 1: Aging Population ---
            \onslide<3->{
                { % Scope start
                    \setbeamercolor{statbox}{bg=orange!10, fg=black}
                    \begin{beamercolorbox}[sep=0.8em, rounded=true, wd=\textwidth]{statbox}
                        \textbf{The "Silver Tsunami"} \\
                        Population aged 65+ will \textbf{DOUBLE} by 2050.
                    \end{beamercolorbox}
                }
            }
            \vspace{0.3cm}

            % --- Stat Box 2: Doctor Shortage ---
            \onslide<4->{
                { % Scope start
                    \setbeamercolor{statbox}{bg=red!10, fg=black}
                    \begin{beamercolorbox}[sep=0.8em, rounded=true, wd=\textwidth]{statbox}
                        \textbf{The Radiologist Shortage} \\
                        Projected gap of \textbf{42,000 specialists} by 2033.
                    \end{beamercolorbox}
                }
            }

            \vspace{0.5cm}
            \onslide<5->{
                \centering \small
                \textit{AI-assisted screening is a necessity, not a luxury.}
            }
        \end{column}
    \end{columns}
\end{frame}

% --- SLIDE 2: VISUALIZING THE PROBLEM (Animated TikZ) ---
\begin{frame}{The Core Technical Flaw: "Data Leakage"}

    Many previous "high-accuracy" studies fail in clinical practice. Why?
    \vspace{0.3cm}
    \centering

    % Define colors locally
    \definecolor{LocalGreen}{RGB}{0, 150, 80}
    \definecolor{LocalBlue}{RGB}{0, 51, 102}

    \resizebox{0.95\textwidth}{!}{%
    \begin{tikzpicture}[
        node distance=1.5cm and 1cm,
        font=\sffamily\footnotesize,
        patient/.style={
            rectangle, rounded corners=5pt, draw=LocalBlue, fill=blue!10,
            minimum height=1.2cm, minimum width=1.5cm, align=center, thick
        },
        slice/.style={
            rectangle, draw=gray, fill=white, thin, minimum width=0.15cm, minimum height=0.6cm
        },
        box/.style={
            rectangle, draw=black, thick, minimum width=3cm, minimum height=1.5cm, rounded corners, align=center
        },
        arrow/.style={->, >=stealth, thick}
    ]
        % --- IMPORTANT: BOUNDING BOX FIX ---
        % This invisible rectangle ensures the resizebox doesn't "jump"
        % when new elements appear. It reserves the full space from Slide 1.
        \path[use as bounding box] (-2.5, -3.5) rectangle (12, 4);

        % === LEFT SIDE (The Problem) ===
        \node<1->[font=\bfseries\small] at (0, 3.5) {\textcolor{red}{THE WRONG WAY} (Slice-based)};
        \node<1->[patient] (p1) at (0, 2) {Patient A\\(MRI Vol)};

        % Slices (Always visible to show context)
        \foreach \x in {-0.6, -0.3, 0, 0.3, 0.6} {
            \node<1->[slice, below=0.1cm of p1, xshift=\x cm] {};
        }

        % Boxes (Wrong Way)
        \node<1->[box, fill=red!5, below left=1.8cm and 0.2cm of p1] (train_bad) {TRAINING SET\\(Slices A1, A3...)};
        \node<1->[box, fill=red!5, below right=1.8cm and 0.2cm of p1] (test_bad) {TESTING SET\\(Slices A2, A4...)};

        % ANIMATION STEP 2: Show the Leakage Arrows & Warning
        \draw<2->[arrow, red, dashed] (0, 1.2) -- (train_bad.north);
        \draw<2->[arrow, red, dashed] (0, 1.2) -- (test_bad.north);
        \node<2->[below=0.2cm of train_bad, text=red, font=\bfseries\scriptsize, align=center, xshift=2cm] {
            \textbf{[!] DATA LEAKAGE}\\Model "memorizes" Patient A anatomy.
        };

        % === RIGHT SIDE (The Solution) ===
        \node<3->[font=\bfseries\small] at (8, 3.5) {\textcolor{LocalGreen}{THE RIGHT WAY} (Subject-wise)};
        \node<3->[patient] (pA) at (6.5, 2) {Patient A};
        \node<3->[patient] (pB) at (9.5, 2) {Patient B};

        % Boxes (Right Way)
        \node<3->[box, fill=green!5, below=1.8cm of pA] (train_good) {TRAINING SET\\(All of A)};
        \node<3->[box, fill=green!5, below=1.8cm of pB] (test_good) {TESTING SET\\(All of B)};

        % ANIMATION STEP 4: Show the Correct Flow & Barrier
        \draw<4->[arrow, LocalGreen, thick] (pA.south) -- (train_good.north);
        \draw<4->[arrow, LocalGreen, thick] (pB.south) -- (test_good.north);

        % Barrier
        \draw<4->[line width=2pt, LocalGreen] (8, 1) -- (8, -2.5);
        \node<4->[fill=white, text=LocalGreen, font=\bfseries\scriptsize, rotate=90] at (8, -0.8) {STRICT BARRIER};

        % Success Message
        \node<4->[below=0.2cm of train_good, text=LocalGreen, font=\bfseries\scriptsize, align=center, xshift=1.5cm] {
            \textbf{[OK] ROBUST}\\Testing on unseen brains.
        };

    \end{tikzpicture}%
    }
\end{frame}